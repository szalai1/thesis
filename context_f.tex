\chapter{Context-aware recommendation}\label{ch:context}
The continuous online presence of users makes possible the recommender systems to 
collect  more details about the users' online behave and put their actions into 
context. Context based recommendation systems use not only the fact of the interaction between 
user and item, but exploit the opportunities of the session like  time series. 

Context can be anything, which usable to differentiate a given interaction of a user-item pair \cite{adomavicius2011context}
from another interaction of the same pair. In some case this context is time, but can be the 
user behave in the session, other users impact, current global trends or the user's or item's
geographical location at the time of the interaction.
\section{Location-aware recommendation}

\section{Data with location information}

\subsection{Data parsing and cleansing}
The dataset we examine was gathered between February 1 2012 and January 1 2013
through Twitter's streamingAPI and was restricted to geo-tagged tweets. The
dataset consists of more that 1.4 billion tweets, including the original
tweets of around 30 million retweets. In the base data set each tweet is a
JSON document. We parsed the JSON files, and created CSV files that archive
only the relevant information for us. Duplicated or unparseable lines have been
removed. In case of a retweet, the JSON document contains information on the
original root tweet, that we added to the set of tweets. Because of the
simplicity of CSV, the data may have information redundancy, but it allows us
to make more robust and reusable codes. Since we wish to work with entities and
attributes (like hashtags and @-mentions) using their timestamp, we organized
the different entities into different files and grouped and sorted them
accordingly. By reason of the size of the data, we run our parser applications
in parallel and after merge the results. That is how we get our main timeline
es, for instance hashtag, URL, mention, user timelines. All timelines contain
not only timestamps and entities, but also the geographical information of the
tweets.
%\subsection{Statistics}

%\section{Using matrix factorization for hashtag recommendation}


%\section{Using recency and popularity functions for hashtag recommendation}

%\section{Music recommendation}
%\subsection{30M music data}


%\subsection{Attribute based recommendation}
%\subsection{Hiden markov}