\chapter{Context-aware recommendation}\label{ch:context}
The continuous online presence of users makes possible the recommender systems to 
collect  more details about the users' online behave and put their actions into 
context. Context based recommendation systems use not only the fact of the interaction between 
user and item, but exploit the opportunities of the session like  time series. 

Context can be anything, which usable to differentiate a given interaction of a user-item pair \cite{adomavicius2011context}
from another interaction of the same pair. In some case this context is time, but can be the 
user behave in the session, other users impact, current global trends or the user's or item's
geographical location at the time of the interaction.

\section{Related work}

\subsection{Location-aware recommendation}

\subsection{Music recommendation using user tags}


%\subsection{Statistics}

%\section{Using matrix factorization for hashtag recommendation}


%\section{Using recency and popularity functions for hashtag recommendation}

%\section{Music recommendation}
%\subsection{30M music data}


%\subsection{Attribute based recommendation}
%\subsection{Hiden markov}
