\chapter{Introduction}
The storage capacity for a given cost is increasing exponentially \cite{wiki:krydr}.
This tendency allows us to store information at low cost. Furthermore, services like the WWW
resulted accelerated communication patterns in a highly connected human society. As a
result, accessing information at no cost become a standard in the 21st century. However,
for a given person, the amount of reachable information is too large to process. In the age of big
data, users of the WWW are flooded by accessible information. Services like search engines
and recommender systems serve to filter for the given user the available information, and
only retrieve relevant information for her. 

Recommender systems are able to solve this problem by filtering items and recommend the 
suitable ones to users. Recommender systems can  also predict user-item interactions and
the  user behavior. These systems are used by online companies/providers/websites to improve
the users satisfaction and to increase the number of their users in the same time. 

Recommender systems also able to use event related information, so called contextual information,
to make their recommendation more accurate. These contxtual informations can anything, which 
distinguish two user item interaction for example geolocation, session, item or user releated informations.

In this thesis, after we explained the basic mathematical background of the recommender systems,
discuss about recommender systems in general and focus on some specific fields of this topic, such as \emph{online} evaluation and \emph{context-based} recommendation. Furthermore, we present some of our result.

The structure of this thesis is the following. In Section~\ref{ch:rec_sys}, we go through the 
evaluation technics, models  and recommendation variants. In Section~\ref{ch:mf}, we explain the 
essential mathematical background of the matrix factorization and its learning process. After that
in Section~\ref{ch:context} we discuss about different type of context-aware recommendation. Finally,
we present out models and expound their performance in context of known baseline algorithms, in 
Section~\ref{ch:res}.